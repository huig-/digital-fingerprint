En los últimos a\~nos la ciberseguridad ha sido una de las ramas más estudiadas en informática, con el gran problema que supone que siempre las defensas se han desarrollado ante los ataques existentes lo que ha suponido que las defensas estaban muy retrasadas con respecto a los ataques. Entre la diversidad de ataques que existen uno de los que ha tenido gran relevancia son los ataques de denegación de servicio, que causan pérdidas económicas muy grandes a las empresas afectadas. A pesar de los avances en cuanto a la defensa y detección de los ataques se refiere, los atacantes buscan nuevas formas más sofisticadas para burlar estas defensas. En mi opinión, debe trabajarse en la modelización del tráfico legítimo de las redes para poder llevar a cabo una detección de anomalías precisa y con un número muy bajo de falsos positivos. Para ello no solamente se debe trabajar en las defensas sino que se deben desarrollar de forma controlada ataques y generar datasets con muestras y dar un marco de valoración efectivo a los investigadores para poder medir la calidad de los sistemas defensivos. Además, las causas principales de que existan los ataques de denegación de servicio son la posibilidad de falsificar la dirección IP origen y la existencia de botnets. Trabajar en implementar en todos los dispositivos de Internet el protocolo seguro IPsec podría dar solución al primero de los problemas, y mejorar la seguridad de cada uno de los equipos que se conectan a Internet también debe ser una prioridad para reducir el riesgo de infección de equipos para conformar botnets. A pesar de estar mencionando varios problemas abiertos y complicados que existen ahora mismo para solucionar este problema, es importante ver que hay muchos desafíos aún que resolver y que la mejora de cada uno sirve para resolver muchos otros problemas existentes.
