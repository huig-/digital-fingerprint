Los ataques de denegación de servicio constituyen una de las amenazas más importantes y extendidas en el ámbito de la ciberseguridad, según datos de \cite{akamai:2016} aumentando un $125\%$ a\~no tras a\~no. Además de crecer en frecuencia, también crece en sofisticación y en ancho de banda. En cuanto a la sofisticación están aprovechando últimamente vulnerabilidades en servidores DNS y en el protocolo NTP (Network Time Protocol), usado para sincronizar fechas y horas entre distintas máquinas de una red. En cuanto al ancho de banda estos ataques han aumentado en volumen, con ataques sostenidos de 200 Gbps\cite{sans}. Sin embargo, muchas empresas no están preparadas para este tipo de ataques e incluso muchas de ellas desconocen esta amenaza (un $40\%$ de las empresas no está preparada para este tipo de ataques\cite{sans}). Estos ataques además tienen muchas veces por objetivo el encubrimiento de delitos como pueden ser transferencias fraudulentas de dinero o el desanonimato\cite{sniper:2014}. \\

Los ataques de denegación de servicio (del inglés \textit{Denial of Service Attacks}) o DoS, tienen como objetivo comprometer la disponibilidad de un servicio agotando sus recursos. Estos ataques pueden intensificarse en ancho de banda al originarse desde diversas máquinas en lugar de solamente una, conociéndose entonces como ataques de denegación de servicio distribuidos (DDoS). Estos ataques distribuidos suelen realizarse utilizando botnets (una red de ordenadores comprometida). Básicamente estos ataques se llevan a cabo mediante dos métodos distintos: mediante inundación que consiste en agotar los recursos mediante un número muy alto de peticiones haciendo que usuarios legítimos no puedan ser atentidos (por ejemplo el ataque ICMP) y mediante vulnerabilidades que consiste en aprovechar fallos (de implementación normalmente) de algún protocolo (por ejemplo la conocida vulnerabilidad \textit{Heartbleed} de la librería OpenSSL)\cite{peng:2007}. \\

El crecimiento de los ataques de denegación de servicio es debido a diversos factores, que se suelen clasificar en función de las motivaciones del mismo. Algunas de ellas son\cite{peng:2007}:
\begin{itemize}
\item \textbf{Económicas:} se basan en reducir el poder de la competencia al inutilizar los recursos. Suele ser bastante frecuente en empresas de videojuegos multijugador online, en los que empresas competidoras tratan de inutilizar los servidores de la otra empresa para ganar consumidores. 
\item \textbf{Creencias:} grupos políticos o culturales que expresan su contrariedad frente a otro tipo de creencias.
\item \textbf{Experimentación:} individuos que tratan de mejorar sus habilidades cometiendo este tipo de ataques, para posteriormente utilizarlos con otro tipo de motivaciones.
\end{itemize} 


El crecimiento y la variedad de estos ataques hacen que sea difícil desarrollar una estrategia defensiva que prevenga estos ataques debido a la gran diversidad de formas que hay para realizarlos. A pesar de esta variedad muchos utilizan técnicas comunes que son la inundación, la amplificación y la reflexión. \\

\textbf{DoS basada en inundación}
Esta es la forma más burda pero también más simple de realizar un ataque de denegación de servicio, se basa en la inyección de grandes volúmenes de tráfico. Estos ataques actúan sobre la capa de red o sobre la capa de aplicación. En el caso de la capa de red los protocolos más explotados son TCP, UDP, ICMP y DNS\cite{douligeris:2004}. Un ejemplo es la inundación SYN del protocolo TCP que se aprovecha del \textit{Handshake} para abrir conexiones falsas en un servidor y que este se quede sin recursos para abrir más conexiones. En el caso de la capa de aplicación el protocolo HTTP es el más explotado. \\

\textbf{DoS basado en reflexión}
La reflexión se basa en intentar ocultar el origen del atacante falsificando la dirección IP origen (IP spoofing). De esta forma se utiliza por ejemplo en el ataque \textit{smurf} provocando que todas las máquinas respondan a la dirección IP origen falsa, en este caso la víctima no sería la dirección IP destino, sería la dirección IP origen. También existen ataques de este tipo sobre VoIP\cite{voip:2008}. \\

\textbf{DoS basado en amplificación}
La amplificación se basa en utilizar terceras partes que generen un paquete de mayor tama\~no, de esta forma, falsificando también la dirección de retorno (también utilizan la reflexión) la victíma recibe paquetes de gran tama\~no originados con un ancho de banda menor. El ejemplo más común de este tipo de ataques se basa en el protocolo DNS, cuyas respuestas pueden tener un tama\~no muy grande comparado con el de la consulta\cite{dns:2013}. 
