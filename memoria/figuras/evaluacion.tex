Uno de los problemas en el área de los ataques de denegación de servicio son los pocos datos públicos y abiertos que existen sobre ataques realizados, incluyendo las trazas. Por esto es muy difícil para la comunidad científica valorar y estimar las estrategias defensivas. Como se explica en \cite{bhatia}, la mayor parte de las colecciones públicas de muestras de ataques carecen de validez por diferentes motivos, ya sea por antig\"uedad o falta de rigor en los procesos de captura. A continuación se van a mencionar y comentar las dos colecciones más usadas:
\begin{itemize}
\item \textbf{KDD'99.} Estas muestras están recogidas del concurso KDDcup del a\~no 1999 e incluye parte de un conjunto de trazas de ataques publicadas por la agencia norteamericana DARPA. Además de ser un dataset antig\"uo, no solamente incluye tráfico de denegación de servicio sino que también incluye otro tipo de ataques lo que le hace perder validez científica. Los datos presentes en las muestras han sido totalmente anonimizados y no se tienen las trazas originales, sino que se tienen expresadas en función de $41$ parámetros.
\item \textbf{CAIDA'07.} Contiene trazas de ataques de inundación como ICMP, SYN y HTTTP en formato reconocible por Wireshark, capturadas en el 2007\cite{caida}. Es el dataset que los científicos consideran más válido a pesar de que tiene el problema de que solamente contiene tráfico de ataque y por lo tanto no puede establecerse un modelo del tráfico legítimo para detectar una anomalía posteriormente. Para realizar el modelo se suele usar la colección CAIDA'08\cite{caida2} que contiene tráfico legítimo del mismo router aunque un tiempo después. 
\end{itemize}
