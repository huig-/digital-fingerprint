Los ataques de denegación de servicio suponen un gran reto para la ciberseguridad puesto que es muy difícil desarrollar un sistema defensivo contra estos. Normalmente los sistemas se clasifican según el momento del ataque en el que actúan: prevención, detección, identificación del origen y mitigación. Todas estas fases son importantes y los sistemas más sofisticados deben tratar cada una de las fases para ser lo más completos posibles. Cada una de estos cuatro ejes han sido tratados con profundidad en la bibliografía y tratan de adaptarse a los nuevos ataques que surgen. 

\section{Prevención}
Los métodos pertenecientes a este grupo actúan antes de que el propio ataque suceda y tratan de por una parte minimizar el da\~no a recibir en caso de que ocurra un ataque y por otra parte disminuir la probabilidad de que ocurra un ataque. A pesar de ser una parte importante de los sistemas de defensa son insuficientes sin las demás partes. En \cite{zargar:2013} se enumeran algunas de los métodos usados en esta fase. 

\section{Detección}
Sin duda la fase más importante de cualquiera de los sistemas defensivos puesto que sin esta fase no habría cabida para la identificación del origen ni para la mitigación. Sin embargo es posiblemente la fase más difícil. Su eficacia se basa en la proporción de ataques reales que son detectados, aunque también deben tener en cuenta los falsos positivos, pues es importante no emitir alertas a menudo ante situaciones legítimas de tráfico. Los falsos positivos además acarrean una dificultad a\~nadida debido al fenómeno conocido como \textit{flash-crowd}, que sucede cuando se produce una acumulación inesperada de accesos a un servidor pero de forma legítima, como fue el caso del mundial del 98\cite{detection:2014}. \\

En cuanto a la detección hay dos paradigmas claramente diferenciados: reconocimiento de firmas y detección de anomalías. El reconocimiento de firmas se basa en mantener una historia de la mayor cantidad de ataques reales conocidos y realizados y comparar el tráfico entrante con esta base de datos en busca de similitudes. Sin embargo la identificación de patrones en el reconocimiento de firmas tiene la gran desventaja de que peque\~nas variaciones de ataques ya existentes tendrán patrones distintos y no serán detectados lo que hace que sea una herramienta poco efectiva en un área donde continuamente evolucionan los ataques y aparecen nuevos. Por otra parte la detección de anomalías se basan en modelar el comportamiento habitual del tráfico de la red e identificar eventos anómalos que difieran de las características del tráfico legítimo. Este método sí que puede detectar nuevos ataques o variaciones de ataques ya existentes pero es más difícil de llevar a cabo un sistema basado en anomalías que uno basado en reconocimiento de firmas. En la actualidad el método que más predomina es la detección de anomalías con gran variedad de técnicas distintas: modelos basados en Markov\cite{markov:2013}, teoría del caos\cite{chaos:2013}, algoritmos genéticos\cite{genetic:2012}, análisis CUSUM\cite{wavelets:2012}, lógica difusa\cite{logic:2013} o el estudio de las variaciones de la entropía aplicando modelos estadísticos\cite{entropy1:2015}\cite{entropy2:2015}.

Muchas de las técnicas que más se utilizan en cuanto a la detección de anomalías es el uso de la entropía\cite{shannon-entropy}\cite{renyi-entropy}, pues la entropía es un indicativo de lo homogéneo/heterogéneo que es un conjunto de datos. En la mayoría de los ataques de denegación de servicio (por lo menos los de inundación) el tráfico atacante tiene unas características muy similares por lo que al haber gran cantidad de este tipo de paquetes se tiene un tráfico muy homogéneo, mientras que en condiciones de tráfico legítimo el tráfico suele ser más heterogéneo. La entropía se aplica como una métrica que permitirá modelar el sistema de diversas formas, algunas de ellas series temporales (como el caso de Holt-Winters o ARIMA), en las que en función del pasado se predecirá el próximo valor de la entropía con un umbral de error, cuando este umbral es traspasado se emitirá una alerta pues se ha detectado una anomalía. 

\section{Identificación del origen}
En esta etapa, que sucede siempre después de haber detectado el ataque, la víctima trata de encontrar la ruta del vector de ataque para descubrir al atacante. Este proceso es muchas veces complicado puesto que el atacante dispone de variados métodos para ocultar su rastro que van desde sencillos procesos de suplantación de identidad hasta incluso hacer uso de redes anónimas. A pesar de no poder encontrar la ubicación real del atacante, acercarse lo máximo posible permite una defensa más efectiva\cite{traceback:2014}. \\

En \cite{traceback2:2014} se presentan distintos métodos y sus diferencias para la identificación del origen, muchos de ellos basados en el marcado de la ruta de los paquetes, ya sea marcando el propio datagrama o almacenando información en dispositivos intermedios.

\section{Mitigación}
La mitigación trata de mitigar el da\~no causado y sus medidas consisten usualmente en el incremento de recursos de sistema o el aumento de la restricción en sistemas de autenticación.
