En este capítulo se tratarán las botnets, uno de los medios más habituales que utilizan los atacantes para realizar ataques de denegación de servicio distribuidos, que además de aumentar el ancho de banda del ataque hace inútil el proceso de identificación del origen.

El término botnet designa a un conjunto de máquinas infectadas (conocidas como bots), que están bajo control de un operador humano (conocido como el botmaster). Los bots son utilizados para llevar a cabo gran variedad de acciones maliciosas y da\~ninas contra sistemas y servicios, incluyendo ataques de denegación de servicio, distribución de spam y phishing. Están dise\~nadas con motivos económicos y normalmente el botmaster pone en alquiler la botnet en servicio de terceros a cambio de remuneraciones económicas\cite{botnet-art}. 

La arquitectura de una botnet como se ha mencionada anteriormente, está compuesta por bots (que viene de robot, pues siguen las instrucciones que les da un operador humano) y por el botmaster, que se comunica con ellos mediante mensajes C\&C (Command and Control), por lo que se deben establecer canales para dicha comunicación, que usualmente están basados en IRC. Normalmente estos bots se adquieren por el botmaster mediante la infección de los equipos utilizando vulnerabilidades que permiten el acceso con superusuario al ordenador comprometido, haciendo que el due\~no del equipo infectado sea un participante pasivo de los actos delictivos\cite{botnet-art}.

\section{Ejemplos de botnets}
En esta sección se presentan algunas de las botnets más importantes descubiertas hasta la fecha. A pesar de que se comente principalmente su función en cuanto al envío de spam por email hay que tener en cuenta que los ataques de denegación de servicio que han podido causar no han sido detectados o no se conseguido identificar su origen, por lo que no hay que deshechar que realizarán esta actividad. Estas botnets y otras vienen comentadas en \cite{botnet-ej}.
\begin{itemize}
\item \textbf{Grum} fue creada en el a\~no 2008, y en cuatro a\~nos se convirtió en responsable de hasta el $26\%$ de todo el spam de Internet por email. En el a\~no 2010 era capaz de emitir hasta 39.9 billones de mensajes por día, convirtiéndola en la botnet más grande descubierta hasta esa fecha.
\item \textbf{ZeroAccess} es una de las botnets más recientes en ser detectada y cerrada. Se estimaba que tenía el control sobre 1.9 millones de ordenadores distribuidos por todo el mundo y su provecho económico se basada en \textit{click fraud} y minería de bitcoins. Fue descubierta por la minería de bitcoins, pues consumía energía suficiente para dar luz a $111000$ casas cada día.
\item \textbf{Windigo} se descubrió en el a\~no 2014 y su nombre viene de un mitológico monstruo caníbal. Habiendo operado durante solamente tres a\~nos, había infectado $10000$ servidores Linux (no ordenadores), permitiendo enviar 35 millones de emails con spam cada día. Curiosamente, mandaba distinto contenido en función del sistema operativo del dispositivo que lo iba a recibir: enviaba malware a Windows, páginas web de citas a usuarios de Mac OS X, y contenido pornográfico a usuarios de iPhone.
\item \textbf{Cutwail} controlaba dos millones de ordenadores en el a\~no 2009, mandando hasta 74 billones de emails con spam cada día (más o menos 1 millón por minuto). En ese momento era el $46.5\%$ del spam mundial total.
\item \textbf{Srizbi} solamente estuvo activo durante un a\~no, a pesar de ser responsales del $60\%$ del spam mundial de 2007 a 2008. Cuando se desarticuló está botnet, el spam mundial cayó un $75\%$.
\end{itemize}

