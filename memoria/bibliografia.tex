\begin{thebibliography}{11}

\bibitem{goljan:2006}
J. Lukas, J. Fridrich, M. Goljan (2006).''Digital camera identification from sensor pattern noise''. \textit{IEEE Transactions on Information Forensics and Security, 1(2), 205-214.}

\bibitem{wiki:db8}
Daubechies wavelet. Available: \url{https://en.wikipedia.org/wiki/Daubechies\_wavelet}

\bibitem{dabov:2007}
K. Dabov, A. Foi, V. Katkovnik \textit{et al} (2007). ''Image denoising by sparse 3-d transform domain collaborative filtering''. \textit{IEEE Trans. Image Process., 16(8), pp. 2080-2095}

\bibitem{gisolf:2013}
F. Gisolf, A. Malgoezar, T. Baar, \textit{et al} (2013). ''Improving source camera identification using a simplified total variation based noise removal algorithm''. \textit{Digital Invest., 10(3), pp. 207-214}

\bibitem{gprnu:2016}
M. Al-Athamneh, F. Kurugollu, D. Crookes, M. Farid (2016). ''Digital video source identification based on green-channel photo response non-uniformity (G-PRNU)''. \textit{Sixth International Conference on Computer Science, Engineering \& Applications}

\bibitem{goljan:2008}
T. Filler, J. Fridrich, M. Goljan (2008). ''Using sensor pattern noise for camera model identification''. \textit{15th International Conference on Image Processing, San Diego, CA, 2008, pp. 1296-1299}

\bibitem{b1:2015}
A. Murat (2015). ''Digital Video Processing, Second Edition''. \textit{Prentice Hall Signal Processing}

\bibitem{b2:2005}
A. Bovik (2005). ''Handbook of Image and Video Processing, Second Edition''. \textit{Academic Press}

\bibitem{b3:2012}
M. Parker, S. Dhanani (2012). ''Digital Video Processing for Engineers''. \textit{Newnes}

\bibitem{wiki:wiener}
Wiener Filter. Available: \url{https://en.wikipedia.org/wiki/Wiener\_filter}

\bibitem{sowmya:2015}
K. Sowmya, H. Chennamma (2015). ''A survey on video forgery detection''. \textit{International Journal of Computer Engineering and Applications, Volume IX}

\bibitem{bestagini:2012}
P. Bestagini, M. Fontani, S. Milani, M. Barni \textit{et al} (2012). An overview on video forensics. \textit{APSIPA Transactions on Signals and Information Processing, V1, pp. 1229-1233}

\bibitem{mondaini:2007}
N. Mondaini, R. Caldelli, A. Piva, M. Barni \textit{et al} (2007). Detection of malevolant changes in digital video for forensic applications. \textit{Proc. of SPIE, Security, Steganoprahy, and Watermarking of Multimedia Contents IX, E.J.D III and P. W. Wong, eds., vol. 6505, no. 1, SPIE, 65050T}

\bibitem{kobayashi:2010}
M. Kobayashi, T. Okabe, Y. Sato (2010). Detecting forgery from static-scene video based on inconsistency in noise levels functions. \textit{IEEE Trans. Info. Forensic Secur., 5(4). pp. 883-892}

\bibitem{conotter:2011}
V. Conotter, J. O'Brien, H. Farid (2011). Exposing digital forgeries in ballistic motion. \textit{IEEE Trans. Info. Forensics Secur., pp 99}

\bibitem{fridrich:1998}
J. Fridrich (1998). \textit{Image watermarking for tamper detection}. \textit{ICIP (2), pp. 404-408}

\bibitem{fridrich:2003}
J. Lukás, J. Fridrich (2003). Estimation of primary quantization matrix in double compressed jpeg images. \textit{Proc. of DFRWS}

\bibitem{milani:2012}
S. Milani, M. Tagliasacchi, M. Tubaro (2012). Discriminating multiple jpeg compression using first digit features. \textit{Proc. of the 37\textsuperscript{th} Int. Conf. on Acoustics, Speech, and Signal Processing (ICASSP), pp. 2253-2256}

\bibitem{wang:2009}
W. Wang, H. Farid (2009). Exposing digital forgeries in video by detecting double quantization. \textit{Proc. 11\textsuperscript{th} ACM Workshop on Multimedia and Security, MM\&Sec '09, ACM, New York, NY, pp. 39-48}

\bibitem{farid:2009}
W. Wang, H. Farid (2009). Exposing Digital Forgeries in Video by Detecting Double MPEG Compression. \textit{MM\&Sec'06 Proc. of the 8th workshop on Multimedia and Security, pp. 37-47}

\bibitem{b4:2012}
K. Sayood (2012). Introduction to Data Compression, 4th Edition. \textit{Morgan Kaufmann}

\bibitem{villalba:2015}
L.J. García Villalba, A. Lucila Sandoval, J. Rosales Corripio (2015). Smartphone image clustering. \textit{Expert Systems with Applications, 42, pp. 1927-1940}

\bibitem{wiki:huffman}
Codificación Huffman. Available: \url{https://en.wikipedia.org/wiki/Huffman\_coding}

\bibitem{chang:2013}
G. Lin, J. Chang (2013). Detection of Frame Duplication Forgery in Videos based on Spatial and Temporal Analysis. \textit{International Journal of Pattern Recognition and Artificial Intelligence}

\bibitem{huang:2014}
Z. Huang, F. Huang, J. Huang. Detection of double compression with the same bit rate in MPEG-2 videos. \textit{IEEE China Summit \& International Conference on Signal and Information Processing (ChinaSIP)}

\bibitem{yao:2017}
H. Yao, S. Song, C. Qin, Z. Tang, X. Liu (2017). Detection of Double-Compressed H.264/AVC Video Incorporating the Features of String of Data Bits and Skip Macroblocks. \textit{Symmetry}

\bibitem{vazquez:2012}
D. Vázquez-Padín, M. Fontani, T. Bianchi, P. Comesa\~na \textit{et al} (2012). Detection of video double encoding with GOP size estimation. \textit{IEEE International Workshop on Information Forensics and Security (WIFS)}

\bibitem{wang:2013}
W. Wang, X. Jiang, S. Wang, T. Sun (2013). Estimation of the primary quantization parameter in MPEG videos. \textit{Visual Communications and Image Processing (VCIP)}

\bibitem{wu:2014}
Y. Wu, X. Jiang, T. Sun, W. Wang (2014). Exposing video inter-frame forgery based on velocity field consistency. \textit{ICASSP, IEEE International Conference on Acoustics, Speech and Signal Processing - Proceedings, pp. 2674-2678}

\bibitem{wiki:grubbs}
Grubb's test. Available: \url{https://en.wikipedia.org/wiki/Grubbs'\_test\_for\_outliers}

\bibitem{wang:2014}
Q. Wang, Z. Li, Z. Zhang, Q. Ma (2014). Video Inter-Frame Forgery Identification Based on Consistency of Correlation Coefficients of Gray Values. \textit{Journal of Computer and Communications, 2, pp. 51-57}

\bibitem{flow:2014}
Q. Wang, Z. Li, Z. Zhang, Q. Ma (2014). Video Inter-Frame Forgery Identification Based on Optical Flow Consistency. \textit{Sensors \& Transducers, 166, pp. 229-234}

\bibitem{fu:2009}
D. Fu, Y. Q. Shi, W. Su (2009). A generalized benfords law for jpeg coefficients and its applications in image forensics. \textit{Proc. of SPIE, Security, Steganography and Watermarking of Multimedia Contents IX, vol. 6505, pp. 39-48} 

\bibitem{reininger:1983}
R. C. Reininger, J. D. Gibson (1983). Distributions of the two dimensional DCT coefficients for images. \textit{IEEE Trans. On Commun., vol. COM-31, pp. 835-839}

\bibitem{eggerton:1986}
J. D. Eggerton, M. D. Srinath (1986). Statistical distribution of image DCT coefficients. \textit{Computer and Electrical Engineering, vol. 12, pp. 137-145}

\bibitem{tariang:2017}
D. B. Tariang, A. Roy, R. S. Chakraborty, R. Naskar (2017). Automated JPEG forgery detection with correlation based location. \textit{Proceedings of the IEEE International Conference on Multimedia and Expo Workshops (ICMEW)}

\bibitem{sun:2013}
X. Jiang, W. Wang, T. Sun, Y. Q. Shi \textit{et al} (2013). Detection of Double Compression in MPEG-4 Videos Based on Markov Statistics. \textit{IEEE Signal Processing Letters}

\bibitem{chen:2009}
W. Chen, Y. Q. Shi (2009). Detection of double MPEG compression based on first digit statistics. \textit{Lect. Notes Comput. Sci. (IWDW 2008), vol. 5450, pp. 16-30}

\bibitem{jia:2018}
S. Jia, Z. Xu, H. Wang, C. Feng, T. Wang (2018). Coarse-to-fine Copy-move Forgery Detection for Video Forensics. \textit{IEEE Access}

\bibitem{kingra:2017}
S. Kingra, N. Aggarwal, R. D. Singh (2017). Video Inter-frame Forgery Detection Approach for Surveillance and Mobile Recorded Videos. \textit{International Journal of Electrical and Computer Engineering (IJECE), vol. 7, no. 2, pp. 831-841}

\bibitem{dong:2010}
Y. Su, J. Xu, B. Dong, J. Zhang (2010). A novel source MPEG-2 video identification algorithm. \textit{International Journal of Pattern Recognition and Artificial Intelligence, vol. 24, no. 8, pp. 1311-1328}

\bibitem{naveen:2016}
S. Naveen, J. A. Reyaz, C. Balan (2016). Video Source Identification. \textit{International Journal of Computer Science and Information Technologies (IJCSIT), vol. 7 (1), pp. 363-366}

\bibitem{chen:2007}
M. Chen, J. Fridrich, M. Goljan, J. Lukás (2007). Source Digital Camcorder Identification Using Sensor Photo Response Non-Uniformity. \textit{Proc. SPIE 6505, Security, Steganography, and Watermarking of Multimedia Contents IX, vol. 6505, 65051G}

\bibitem{yahaya:2012}
S. Yahaya, A. TS Ho, A. A. Wahab (2012). Advanced video camera identification using conditional probability features. \textit{Proc. of the IET Conference on Image Processing, pp. 1-5}

\bibitem{su:2009}
Y. Su, J. Xu, B. Dong (2009). A source video identification algorithm based on motion vectors. \textit{Proc. of the Second International Workshop on Computer Science and Engineering, vol. 2, pp. 312-316}

\bibitem{wiki:dendrogram}
Dendrogram. Available \url{https://en.wikipedia.org/wiki/Dendrogram}

\bibitem{gap:2001}
R. Tibshirani, G. Walther, T. Hastie (2001). Estimating the number of clusters in a dataset via de Gap statistic. \textit{Journal of the Royal Statistical Society Series B (Statistical Methodology), vol. 63 (2), pp. 411-423}

\bibitem{ana:2015}
A. L. Sandoval, L. J. García Villalba, D. M. Arenas, J. Rosales \textit{et al} (2015). Smartphone Image acquisition Forensics using Sensor Fingerprint. \textit{IET Computer Vision, vol. 9 (5), pp. 723-831}

\bibitem{jain:1996}
A. K. Jain, A. Vailaya (1996). Image retrieval using color and shape. \textit{Pattern recognition, vol. 29 (8), pp. 1233-1244}

\bibitem{jeong:2001}
S. Jeong (2001). Histogram-Based Color Image Retrieval. \textit{Psych221/EE362 Project Report, Stanford University}

\bibitem{zhang:2017}
M. Zhang, L. Tian, C. Li (2017). Key frame extraction based on entropy difference and perceptual hash. \textit{IEEE International Symposium on Multimedia}

\end{thebibliography}
